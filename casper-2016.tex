%%%%%%%%%%%%%%%%%%%%%%%%%%%%%%%%%%%%%%%%%%%%%%%%%%%%%%%%%%%%%%%%%%%%%%%%%%%%
%% Trim Size : 11in x 8.5in
%% Text Area : 9.6in (include Runningheads) x 7in
%% ws-jai.tex, 26 April 2012
%% Tex file to use with ws-jai.cls written in Latex2E.
%% The content, structure, format and layout of this style file is the
%% property of World Scientific Publishing Co. Pte. Ltd.
%%%%%%%%%%%%%%%%%%%%%%%%%%%%%%%%%%%%%%%%%%%%%%%%%%%%%%%%%%%%%%%%%%%%%%%%%%%%
%%

%\documentclass[draft]{ws-jai}
\documentclass{ws-jai}
\usepackage[flushleft]{threeparttable}
\usepackage{tabulary}

\begin{document}

\catchline{}{}{}{}{} % Publisher's Area please ignore

\markboth{Jack Hickish, CASPER gang, etc.}{The Collaboration for Astronomy Signal Processing and Electronics Research in 2016.}

\title{The Collaboration for Astronomy Signal Processing and Electronics Research in 2016}

%\author{First Author$^\dagger$, Second Author$^\ddagger$, Third Author$^\ddagger$ and Fourth Author$^\S$}
\author{Jack Hickish$^\dagger$, Others$^\ddagger$}

\address{
$^\dagger$Radio Astronomy Laboratory, UC Berkeley, Berkeley, CA 94720, USA, jackh@astro.berkeley.edu\\
$^\ddagger$Group, Company, Address, City, State ZIP/Zone, Country\\
$^\S$Group, Company, Address, City, State ZIP/Zone, Country, fauthor@company.com
}

\maketitle

\corres{$^\dagger$Jack Hickish}

\begin{history}
\received{(to be inserted by publisher)};
\revised{(to be inserted by publisher)};
\accepted{(to be inserted by publisher)};
\end{history}

\begin{abstract}

The Collaboration for Astronomy Signal Processing and Electronics Research
(CASPER) has been working for a decade to reduce the time and cost of designing,
building and deploying new digital radio astronomy instruments.  Today,
CASPER-designed hardware powers some ?? scientific instruments worldwide, and
are used by scientists and engineers at ?? academic institutions.  In this paper
we summarize the current offerings of the CASPER collaboration, focussing on
currently-available and next-generation hardware.  We describe the ongoing state
of software development, as CASPER looks to support an ever-increasing selection
of off-the-shelf digital signal processing platforms.

\end{abstract}

\keywords{CASPER, digital signal processing, radio astronomy, instrumentation}


\section{Introduction}

Since the first digital instrument used in radio astronomy \citep{Weinreb} we
have seen a growing adoption of digital processing hardware as the foundation on
which radio telescopes are built.  Today, CPUs, GPUs, FPGAs and ASICs power
almost all of the world's radio telescopes, and our ability to do science has
become inextricably linked with our ability to perform digital computation.
With the capability of digital processing hardware scaling exponentially with
Moore's law, the ability to leverage current technology by reducing the
design-time of new instruments is critical in effective deployments of new radio
astronomy instruments.

The Collaboration for Astronomy Signal Processing and Electronics Research
(CASPER) puts \emph{time-to-science}, the time between conception of an
instrument and its deployment, as a central figure of merit in instrument
design. CASPER works to minimize time-to-science by working to develop and
support open-source, general-purpose hardware, software libraries and
programming tools which allow rapid instrument design, and straightforward
upgrade cycles.

With CASPER hardware and software now powering some ?? radio astronomy
instruments worldwide (see Table~\ref{table:casper-instruments}) including some
of the largest, most advanced telescopes ever built, such as the upcoming
MeerKat Array \citep{MeerKAT}, it is appropriate to document the state of the
Collaboration.

In this paper, we first summarize the design philosophy of CASPER in
Section~\ref{sec:CASPER-philosophy}.  In Section~\ref{sec:Hardware} we describe
currently available CASPER hardware offerings, including the range of digitizers
developed and supported by CASPER. Key to CASPER's success are the firmware
libraries and programming infrastructure provided by the collaboration, which we
overview in Section~\ref{sec:Software}.  In Section~\ref{sec:Deployments} we
document the extensive and wide-ranging applications to which CASPER hardware
and design-tools have been applied. Finally, we describe the future direction of
and challenges faced by the CASPER collaboration in Section~\ref{sec:Future},
with concluding remarks in Section~\ref{sec:Conclusions}.

\section{The CASPER Philosophy} \label{sec:CASPER-philosophy}

%% Jason Manley's PHD has a good section on Philosophy and Ethernet.

\subsection{Computing by the yard}

\subsection{Ethernet}

\subsubsection{Multicast \& Commensal Instruments}


\section{CASPER Hardware} \label{sec:Hardware}

\subsection{FPGA platforms}

\subsubsection{ROACH1}

%% Wesley New

\subsubsection{ROACH2}

%% Wesley New

\subsubsection{SKARAB}

%% Adam Isaacson

\subsubsection{SNAP}

\subsection{ADCs \& DACs}


\section{CASPER Software \& Programming Tools} \label{sec:Software}

\subsection{The CASPER Toolflow}

%% Wesley New

\subsection{JASPER Toolflow}

%% Adam Isaacson / Jack Hickish

\subsection{CASPER DSP Libraries}

%% Andrew Martens + anyone else keen to contribute

\emph{Just Another Signal Processing EnviRonment}


\section{CASPER Deployments} \label{sec:Deployments}

\subsection{MeerKAT}

%% Ask Ruby van Rooyen

\begin{table}
\begin{center}
\begin{tabulary}{0.8\textwidth}{LLLLL}
\hline \hline
\textbf{Instrument}    &  \textbf{Description}   &  \textbf{Hardware}      & \textbf{Functions} & \textbf{Date}   \\

\hline
VEGAS (VErsatile GBT Astronomical Spectrometer)                &  
Spectrometer for the Green Bank Telescope, with 10GHz BW 
(one dual-pol input) or 1.25GHz BW (eight dual-pol inputs).  
Wideband modes and narrowband modes (eight digitally tunes 
sub-bands within the 1.25GHz BW).                              & 
Eight ROACH2 boards, 16 ADC5G boards, 16 SFP+ mezzanine boards & 
digitization \& channelization in wideband modes.  
Digitization and sub-banding in narrow band modes.             &
2016                                                           \\ 

\hline 
Green Bank Transient Spectrometer                              &
Spectrometer for the Green Bank 20m antenna, 
with 500MHz BW (one dual-pol input), and 2048 channels         &
One ROACH1 board, two ADC083000 boards                         &
Digitization \& channelization                                 &
2014                                                           \\

\hline 
FLAG (Focal L-band Array for the GBT) Beamformer               &
19 element phased array feed system, 150 MHz instantanious bw  & 
5 ROACH2-R2 boards, 5 SFP+ mezzanine boards                    & 
Channelization \& digital sideband separation                  &
2016                                                           \\

\hline 
SkyNet Spectrometer                                            &
Spectrometer for the 20 m telescope                            & 
ROACH                                                          & 
Channelization                                                 &
20xx                                                           \\

\hline 
GB receiver Lab Spectrometer                                   &
nnn                                                            & 
ROACH                                                          & 
Channelization                                                 &
20xx                                                           \\

\hline 
GUPPI                                                          &
nnn                                                            & 
n                                                              & 
nn                                                             &
20xx                                                           \\

\hline 
WUPPI                                                          &
nnn                                                            & 
n                                                              & 
nn                                                             &
20xx                                                           \\

\hline 
PUPPI                                                          &
nnn                                                            & 
n                                                              & 
nn                                                             &
20xx                                                           \\

\hline 
DIBAS                                                          &
nnn                                                            & 
n                                                              & 
nn                                                             &
20xx                                                           \\

\hline 
FAST VLBE                                                      &
nnn                                                            & 
n                                                              & 
nn                                                             &
20xx                                                           \\

\hline \hline
\end{tabulary}
\caption{CASPER deployments}
\end{center}
\end{table}


\section{Future Directions \& Challenges} \label{sec:Future}

\subsection{Hardware Design Challenges}

\subsubsection{Timing Closure}

\subsubsection{High speed memories}

%% HMC 

\subsection{Support of off-the-shelf hardware}

\subsection{CPU/GPU programming/data-transport}

\subsection{Design re-use}

\subsection{Observatory Integration}


\section{Conclusions} \label{sec:Conclusions}


\bibliographystyle{ws-jai} \bibliography{casper-2016}

\end{document} 
